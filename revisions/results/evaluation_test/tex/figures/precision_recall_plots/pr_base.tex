\usepackage{pgfplots}
\usepackage{pgfplotstable}
\usepackage{xstring}

\usepgfplotslibrary{colorbrewer}
\usepgfplotslibrary{groupplots}
\usepgfplotslibrary{fillbetween}

%% Use Helvetica as base font throughout the paper. This should be
%% included in all plots.
\usepackage[scaled]{helvet}
\renewcommand\familydefault{\sfdefault}
\usepackage[T1]{fontenc}

% Set font:
%\usepackage{fontspec}
%\setmainfont{MinionPro}[
%    Extension = .otf ,
%    UprightFont = *-Regular ,
%    BoldFont = *-Bold ,
%    ItalicFont = *-It ,
%    BoldItalicFont = *-BoldIt
%    ]


% Controls scaling
\usepackage[scale=0.95]{sourcecodepro}

\pgfplotsset{compat=1.17}

% Generic plotting styles for the ROC plots. Can be adjusted to fit
% other scenarios.
\pgfplotsset{%
  roc/.style = {%
    axis x line*      = bottom,
    axis y line*      = left,
    enlargelimits     = false,
    tick align        = outside,
    legend cell align = left,
    legend pos        = south west,
    legend columns    = 1,
    legend style      = {%
      draw         = none,
      font         = \fontsize{6}{3}\selectfont\tt, %6
    },
    legend image post style = {%
      scale = 0.1, %0.20
    },
    %
    grid              = major,
    major grid style  = {%
      lightgray,
      thin
    },
    % Use main font instead of employing a maths font now;
    % this is more consistent.
    tick label style = {
      /pgf/number format/assume math mode = true,
      /pgf/number format/precision        = 2,
    },
    %
    thick,
    %
    xlabel             = {Sensitivity},
    ylabel             = {PPV},
    try min ticks      = 5,
    ymin               = 0,
    ymax               = 1,
    %
    height             = 10cm, %6,7
    width              = 10cm, %6,7
    unit vector ratio* = 1 1 1,
    %
    cycle list/Dark2, %Set1
    %
    /tikz/font = {\small}, %scriptsize
    % Controls the precision of floating point numbers for all other
    % places of the plot.
    /pgf/number format/.cd,
      fixed,
      zerofill,
      precision = 3
  }
}

\newcommand{\makeprplot}[4]{%
    % 1: model name 
    % 2: train dataset (s)
    % 3: eval dataset 
    % 4: model display name
    \addplot+[
        forget plot,
        no marks,
        thin,
        fill opacity = 0.0,
        draw opacity = 0.2,
        name path    = upper
    ]
        table[
            col sep = comma,
            x       = Sensitivity_mean,
            y expr  = \thisrow{PPV_mean} + \thisrow{PPV_std}
        ]
            {../../data/precision_recall/raw_precision_recall_data_#1_#2_#3.csv};
    %
    \addplot+[
        forget plot,
        no marks,
        thin,
        fill opacity = 0.0,
        draw opacity = 0.2,
        name path    = lower
    ]
        table[
            col sep = comma,
            x       = Sensitivity_mean,
            y expr  = \thisrow{PPV_mean} - \thisrow{PPV_std}
        ]
            {../../data/precision_recall/raw_precision_recall_data_#1_#2_#3.csv};
    %
    \addplot+[
        forget plot,
        fill opacity = 0.1,
        draw opacity = 0.0,
    ]
        fill between [of=upper and lower];

    \addplot+[no marks, very thick]
        table[col sep=comma, x=Sensitivity_mean, y=PPV_mean] 
            {../../data/precision_recall/raw_precision_recall_data_#1_#2_#3.csv};
    %
    % Get AUC numbers:
    \pgfplotstableread[%
      col sep = comma, display columns/setting/.style=string type]{../../data/precision_recall/raw_precision_recall_data_#1_#2_#3.csv}\table 
    %
    \pgfplotstablegetelem{0}{auprc_mean}\of{\table}
    % We can also change the units here, in case this is of interest.
    %
    %   \pgfmathparse{100 * \pgfplotsretval}
    %
    \pgfmathparse{\pgfplotsretval}
    \edef\aucmean{\pgfmathresult}
    \pgfplotstablegetelem{0}{auprc_std}\of{\table}
    \pgfmathparse{\pgfplotsretval}
    \edef\aucstd{\pgfmathresult}
    %
    \StrLen{#4,}[\tmplen]
    \edef\temp{\noexpand\addlegendentry{%
        % Get length of label string; we need this to pad the string to
        % the right length.
        #4, \noexpand\StrGobbleRight{\space\space\space\space\space\space}{\tmplen} AUPRC = \noexpand\pgfmathprintnumber[assume math mode = true]{\aucmean}%
        $\pm$%
        \noexpand\pgfmathprintnumber[assume math mode = true]{\aucstd}%
      };%
    }
    \temp
}
